% --------------------------------------------------------------
% This is all preamble stuff that you don't have to worry about.
% Head down to where it says "Start here"
% --------------------------------------------------------------
 
\documentclass[12pt]{article}
\usepackage{graphicx}
\graphicspath{ {./Images/} }
\usepackage[margin=1in]{geometry} 
\usepackage{amsmath,amsthm,amssymb}
\usepackage[margin=1in]{geometry} 
\usepackage{amsmath,amsthm,amssymb}
% \usepackage[spanish]{babel} %Castellanización
\usepackage{tabularx}
\usepackage[T1]{fontenc} %escribe lo del teclado
\usepackage[utf8]{inputenc} %Reconoce algunos símbolos
\usepackage{lmodern} %optimiza algunas fuentes
\usepackage{graphicx}
\graphicspath{ {images/} }
\usepackage{hyperref} % Uso de links
\usepackage{cite}
\usepackage{ifthen}
\usepackage{tikz-qtree}
\usepackage{float}
\usepackage{pgfplots}
\floatstyle{plaintop}
\restylefloat{table}
\usepackage[tableposition=top]{caption}
\usepackage{varwidth}
 \geometry{
 a4paper,
 total={170mm,257mm},
 left=20mm,
 top=15mm,
 bottom=19mm,
 }
 
 \usepackage{siunitx}
\sisetup{per=slash, load=abbr}

    % GRAPHICS
\usepackage{tikz}
\usepackage{pgfplots}
\pgfplotsset{width=7cm,compat=1.3}

 
 
\newcommand{\N}{\mathbb{N}}
\newcommand{\Z}{\mathbb{Z}}
 
\newenvironment{theorem}[2][Theorem]{\begin{trivlist}
\item[\hskip \labelsep {\bfseries #1}\hskip \labelsep {\bfseries #2.}]}{\end{trivlist}}
\newenvironment{lemma}[2][Lemma]{\begin{trivlist}
\item[\hskip \labelsep {\bfseries #1}\hskip \labelsep {\bfseries #2.}]}{\end{trivlist}}
\newenvironment{exercise}[2][Exercise]{\begin{trivlist}
\item[\hskip \labelsep {\bfseries #1}\hskip \labelsep {\bfseries #2.}]}{\end{trivlist}}
\newenvironment{problem}[2][Problem]{\begin{trivlist}
\item[\hskip \labelsep {\bfseries #1}\hskip \labelsep {\bfseries #2.}]}{\end{trivlist}}
\newenvironment{question}[2][Question]{\begin{trivlist}
\item[\hskip \labelsep {\bfseries #1}\hskip \labelsep {\bfseries #2.}]}{\end{trivlist}}
\newenvironment{corollary}[2][Corollary]{\begin{trivlist}
\item[\hskip \labelsep {\bfseries #1}\hskip \labelsep {\bfseries #2.}]}{\end{trivlist}}

\newenvironment{solution}{\begin{proof}[Solution]}{\end{proof}}

%quiggly arrow things
\newcommand{\squig}{$\scriptsize$\sim$\normalsize$\!}
\newcommand{\lsquigend}{$\scriptsize$\lhd\!$\normalsize$}
\newcommand{\rsquigend}{$\scriptsize\rule{.1ex}{0ex}$\rhd$\normalsize$}


\newcounter{index}

\newcommand\squigs[1]{%
  \setcounter{index}{0}%
  \whiledo {\value{index}< #1}
  {\addtocounter{index}{1}\squig}
}

\newcommand\rsquigarrow[2]{$
  \setbox0\hbox{$\squigs{#2}\rsquigend$}%
  \tiny$%
  \!\!\!\!\begin{array}{c}%
  \mathrm{#1}\\%
  \usebox0%
  \end{array}%
  $\normalsize$\!\!%
}

\newcommand\lsquigarrow[2]{$
  \setbox0\hbox{$\lsquigend\squigs{#2}$}%
  \tiny$%
  \!\!\!\!\begin{array}{c}%
  \mathrm{#1}\\%
  \usebox0%
  \end{array}%
  $\normalsize$\!\!%
}
 
 
 
\usepackage{listings}
\usepackage{xcolor}
 
\definecolor{codegreen}{rgb}{0,0.6,0}
\definecolor{codegray}{rgb}{0.5,0.5,0.5}
\definecolor{codepurple}{rgb}{0.58,0,0.82}
\definecolor{backcolour}{rgb}{0.95,0.95,0.92}
 
\lstdefinestyle{mystyle}{
    backgroundcolor=\color{backcolour},   
    commentstyle=\color{codegreen},
    keywordstyle=\color{magenta},
    numberstyle=\tiny\color{codegray},
    stringstyle=\color{codepurple},
    basicstyle=\ttfamily\footnotesize,
    breakatwhitespace=false,         
    breaklines=true,                 
    captionpos=b,                    
    keepspaces=true,                 
    numbers=left,                    
    numbersep=5pt,                  
    showspaces=false,                
    showstringspaces=false,
    showtabs=false,                  
    tabsize=2
}
 
\lstset{style=mystyle}
 

\begin{document}

 
% --------------------------------------------------------------
%                         Start here
% --------------------------------------------------------------

\begin{titlepage} % Suppresses displaying the page number on the title page and the subsequent page counts as page 1
	\newcommand{\HRule}{\rule{\linewidth}{0.5mm}} % Defines a new command for horizontal lines, change thickness here
	
	\center % Centre everything on the page
	
	%------------------------------------------------
	%	Headings
	%------------------------------------------------
	
	\textsc{\LARGE University of Western Australia}\\[1.5cm] % Main heading such as the name of your university/college
	
	\textsc{\Large CITS3003}\\[0.5cm] % Major heading such as course name
	
	\textsc{\large Graphics and Animation}\\[0.5cm] % Minor heading such as course title
	
	%------------------------------------------------
	%	Title
	%------------------------------------------------
	
	\HRule\\[0.4cm]
	
	{\huge\bfseries Assignment 1: Report}\\[0.4cm] % Title of your document
	
	\HRule\\[1.5cm]
	
	%------------------------------------------------
	%	Author(s)
	%------------------------------------------------
	
	\begin{minipage}{0.4\textwidth}
		\begin{flushleft}
			\large
			\textit{Author}\\
			David \textsc{Adams 22497769} % Author's name
		\end{flushleft}
	\end{minipage}
	~
	\begin{minipage}{0.4\textwidth}
		\begin{flushright}
			\large
			\textit{Date}\\
			 \today % Author's name 
		\end{flushright}
	\end{minipage}
	
	

% \vspace{18 mm}
% %	\vfill % Position the date 3/4 down the remaining page
	
% 	{\large\today} % Date, change the \today to a set date if you want to be precise
	
	%------------------------------------------------
	%	Logo
	%------------------------------------------------
% 	  \vfill\vfill  
	%   \begin{figure}[h]
	%       \centering
	%       \includegraphics{tamagotchi_hive.png}
	%       \caption{Sourced from XKCD\cite{xkcd}}
	%       \label{fig:my_label}
	%   \end{figure}
% 	  \includegraphics{Images/seashell.png}
% 	  \caption{Hello world}
	

	 
	%----------------------------------------------------------------------------------------
	
% 	\vfill % Push the date up 1/4 of the remaining page
	
\end{titlepage}

\tableofcontents
\newpage

\section{Part A}

Was a simple implementation, due to openGL's coordinate system as seen in \ref{fig:1}
the rotation about the \(X\) axis (vertical rotation) will be dealt with by \textit{camUpandOverDeg}
and rotation about the \(Y\) axis (horizomtal rotation) corresponds to \textit{camRotSidewaysDeg}. These rotations
are combined (code-wise) in the opposite way to the effect that is seen. specifically an object is Rotated about the \(Y\) axis first then about the \(X\) 
axis then finally translated. The results of these transformations (without other parts implemented) can be seen in 
\ref{fig:2}

\begin{figure}[]
  \centering
  \includegraphics[scale=0.7]{coord.png}
  \caption{}
  \label{fig:1}
\end{figure}

\begin{figure}
  \centering
  \includegraphics[scale=0.5]{partA.png}
  \caption{}
  \label{fig:2}
\end{figure}



\section{Part B}

\section{Part C}

\section{Part D}

\section{Part E}

\section{Part F}

\section{Part G}

\section{Part H}

\section{Part I}

\section{Part J}

\begin{lstlisting}[language=c, caption=Distributed example]
    for k in vertices:
        if k == local:
            broadcast dist[k to other nodes
        else:
            receive dist[k] from broadcast
        for j in vertices:
            for i in vertices:
                dist[i][j] = min(dist[i][j], dist[i][k] + dist[k][j])
\end{lstlisting}

 \bibliographystyle{unsrt}
 \bibliography{ref.bib}

 
\end{document}